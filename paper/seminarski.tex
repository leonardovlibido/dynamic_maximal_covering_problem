% !TEX encoding = UTF-8 Unicode

\documentclass[a4paper]{article}

\usepackage{color}
\usepackage{url}
\usepackage[T2A]{fontenc} % enable Cyrillic fonts
\usepackage[utf8]{inputenc} % make weird characters work
\usepackage{graphicx}
\usepackage{listings}

\usepackage[english,serbian]{babel}

\usepackage[unicode]{hyperref}
\hypersetup{colorlinks,citecolor=green,filecolor=green,linkcolor=blue,urlcolor=blue}

%\newtheorem{primer}{Пример}[section] %ćirilični primer
\newtheorem{primer}{Primer}[section]

\begin{document}

\title{Lokalna pretraga za dninamički lokacijski problem maksimalnog pokrivanja\\ \small{Seminarski rad u okviru kursa\\Naučno izračunavanje\\ Matematički fakultet}}

\author{Rastko Đorđević, 1091/2017\\ rastko\_djordjevic@matf.bg.ac.rs}
\maketitle

\abstract{
Dinamički lokacijski problem maksimalnog pokrivanja\textit{(eng. Dynamic maximal covering location problem)} na velikoj količini podataka je interesantna oblast istraživanja. U ovom radu će biti opisan problem, metoda lokalne pretrage, i implementacija metode za dati problem u programskom jeziku Python.}

\tableofcontents

\newpage



\section{Postavka problema}

Detaljan opis problema se može naći radu\cite{main_paper}.

\section{Lokalna pretraga}

Pseudo-kod algoritma:
\begin{verbatim}
Lokalna pretraga()
Ulaz:

Izlaz:

begin
end
\end{verbatim}


\section{Implementacija}

Algoritam je implementiran u programskom jeziku Python. 

%\begin{figure}
%	\centering
%	\includegraphics[scale=0.5]{resources/execution_comparison.png}
%	\caption{Poređenje efikasnosti}
%	\label{fig1}
%\end{figure}
\section{Primene}
Ima li primena za ovo?

\addcontentsline{toc}{section}{Literatura}
\appendix
\bibliography{seminarski} 
\bibliographystyle{plain}

\end{document}
