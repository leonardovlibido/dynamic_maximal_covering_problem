% !TEX encoding = UTF-8 Unicode

\documentclass[a4paper]{article}

\usepackage{color}
\usepackage{url}
\usepackage[T2A]{fontenc} % enable Cyrillic fonts
\usepackage[utf8]{inputenc} % make weird characters work
\usepackage{graphicx}
\usepackage{listings}

\usepackage[english,serbian]{babel}

\usepackage[unicode]{hyperref}
\hypersetup{colorlinks,citecolor=green,filecolor=green,linkcolor=blue,urlcolor=blue}

%\newtheorem{primer}{Пример}[section] %ćirilični primer
\newtheorem{primer}{Primer}[section]

\begin{document}

\title{Lokalna pretraga za dninamički lokacijski problem maksimalnog pokrivanja\\ \small{Seminarski rad u okviru kursa\\Naučno izračunavanje\\ Matematički fakultet}}

\author{Rastko Đorđević, 1091/2017\\ rastko\_djordjevic@matf.bg.ac.rs}
\maketitle

\abstract{
Dinamički lokacijski problem maksimalnog pokrivanja\textit{(eng. Dynamic maximal covering location problem)} na velikoj količini podataka je interesantna oblast istraživanja. U ovom radu će biti opisan problem, metoda lokalne pretrage, i implementacija metode za dati problem u programskom jeziku Python.}

\tableofcontents

\newpage



\section{Postavka problema}

Detaljan opis problema se može naći radu\cite{main_paper}.

\section{Lokalna pretraga}

Lokalna pretraga pripada grupi S-metaheuristika, koje se zasnivaju na poboljšavanju vrednosti jednog
rešenja. Na početku algoritma se proizvoljno ili na neki drugi način generiše početno rešenje i izračuna vrednost njegove funkcije cilja. Vrednost najboljeg rešenja se najpre inicijalizuje na vrednost početnog. Zatim se
algoritam ponavlja kroz nekoliko iteracija. U svakom koraku se razmatra rešenje u okolini trenutnog. U zavisnosti od toga kako se definiše okolina mogu se dobiti dobri ilii loši rezultati. Ukoliko je vrednost njegove funkcije cilja bolja od vrednosti funkcije cilja trenutnog rešenja, ažurira se trenutno
rešenje. Takođe se, po potrebi, ažurira i vrednost najboljeg dostignutog rešenja. Algoritam se ponavlja
dok nije ispunjen kriterijum zaustavljanja. Kriterijum zaustavljanja može biti, na primer, dostignut maksimalan broj iteracija, dostignut maksimalan broj ponavljanja najboljeg rešenja, ukupno vreme izvršavanja,
itd. Lokalna pretraga se može prikazati sledećim pseudokodom


Pseudo-kod algoritma:
\begin{verbatim}
Lokalna pretraga()
Ulaz:

Izlaz:

begin
	Generisati početno rešenje s
	Inicijalizovati vrednost najboljeg rešenja f* <= f(s)
	dok nije ispunjen izabrani kriterijum zaustavljanja do
		izabrati proizvoljno rešenje s' u okolini s koje je validno
		ako f(s') < f(s) onda
			s <= s'
		ako f(s') < f* onda
			f* <= f(s')	
end
\end{verbatim}

U datoj verziji lokalne pretrage se vrši problem minimizacije, dok je u našem problemu potrebno maksimizovati ciljnu funkciju. To ćemo rešiti tako što ćemo prihvatati ona rešenja čija je vrednost funkcije cilja veća. Osnovna mana lokalne pretrage je u tome što ne nalazi globalna rešenja, i što kada uđe u lokanlni optimum iz njega ne može da izađe. Ovo se može poboljšati korišćenjem neke naprednije tehnike koja istražuje i rešenja koja nisu trenutno najbolja u nadi da se negde dalje krije bolje rešenje.

\section{Implementacija}

Algoritam je implementiran u programskom jeziku Python. 

%\begin{figure}
%	\centering
%	\includegraphics[scale=0.5]{resources/execution_comparison.png}
%	\caption{Poređenje efikasnosti}
%	\label{fig1}
%\end{figure}
\section{Primene}
Lokacijski problem maksimalnog pokrivanja je klasičan model koji je našao široku primenu u raznim oblastima nauke i privrede. Dinamička verzija ovog problema je relativno nova i još istraživanja je potrebno da bi došlo do njenih primena u većem broju situacija. Ovaj model nije neuobičajen u praksi i primenljiv je u mnogim slučajevima. Na primer može se koristiti za postavku policijskih patrola \cite{police}, premeštanje ambulanti \cite{ambulance} ili realokacija mesta za prvu pomoć kod prirodnih katastrofa.

\addcontentsline{toc}{section}{Literatura}
\appendix
\bibliography{seminarski} 
\bibliographystyle{plain}

\end{document}
