% !TEX encoding = UTF-8 Unicode

\documentclass[a4paper]{article}

\usepackage{color}
\usepackage{url}
\usepackage[T2A]{fontenc} % enable Cyrillic fonts
\usepackage[utf8]{inputenc} % make weird characters work
\usepackage{graphicx}
\usepackage{listings}

\usepackage[english,serbian]{babel}

\usepackage[unicode]{hyperref}
\hypersetup{colorlinks,citecolor=green,filecolor=green,linkcolor=blue,urlcolor=blue}

%\newtheorem{primer}{Пример}[section] %ćirilični primer
\newtheorem{primer}{Primer}[section]

\begin{document}

\title{Lokalna pretraga za dninamički lokacijski problem maksimalnog pokrivanja\\ \small{Seminarski rad u okviru kursa\\Naučno izračunavanje\\ Matematički fakultet}}

\author{Rastko Đorđević, 1091/2017\\ rastko\_djordjevic@matf.bg.ac.rs}
\maketitle

\abstract{
Dinamički lokacijski problem maksimalnog pokrivanja\textit{(eng. Dynamic maximal covering location problem)} na velikoj količini podataka je interesantna oblast istraživanja. U ovom radu će biti opisan problem, metoda lokalne pretrage, i implementacija metode za dati problem u programskom jeziku Python.}

\tableofcontents

\newpage



\section{Postavka problema}

Postavka problema koja će biti objašnjena u nastavku je inspirisana radom\cite{main_paper}. Dobro poznati tip lokacijskih problema koje je izučavan od samog početka oblasti je lokacijski problem pokrivanja u kojem je cilj prekriti skup zahteva uz određeni cilj koji treba optimizovati. Verzija koja će biti razmatrana u ostatku rada je lokacijski problem maksimalnog pokrivanja u kojem je cilj pokriti maksimalan broj zahteva sa poznatim brojem objekata. Pored toga problem će biti još zakomplikovan uvođenjem više perioda.  U takvom problemu cilj je pronaći optimalnu lokaciju za p objekata u m različitih perioda.

\subsection{Matematička definicija}

Kako bismo predstavili DMCLP neophodno je postaviti neke pretpostavke. U daljem radu biće podrazumevano da se koristi diskretni tip problema. Takođe proizvoljni čvor se računa kao pokriven ako se objekat nalazi u dozvoljenoj udaljenosti od njega. Dodatno, skup zahtevnih čvorova je jedan skupu potencijalnih lokacija objekata. Svakom čvoru dodeljujemo vrednost u svakom periodu koja predstavlja težinu koju treba maksimizovati. Ovo možemo gledati kao potražnju datog čvora. Iako je poznat predodređen broj ukupnih objekata koje treba dodeliti ne postoji uslov o tome koliko objekata treba dodeliti u kom periodu.

U nastavku će biti dat skup parametara i promenljivih koje treba optimizovati kao i funkcija koja se maksimizuje.
\begin{description}
\item[$i, I$]  - Indeks i skup čvorova potražnje
\item[$j, J$]  - Indeks i skup čvorova potencijalnih objekata
\item[$a_{it}$] - Potražnja u čvoru $i$ u periodu t
\item[$d_{ij}$] - Najkraća distanca između čvora $i$ i objekta $j$
\item[$S$] - Maksimalna distanca koja određuje da li je čvor pokriven 
\item[$N_{i}$] - Skup čvorova takvih da postoji bar jedan objekat koji je na distanci manjoj od $S$ od njih
\item[$p$] - broj objekata koje treba postaviti
\item[$X_{jt}$] - Binarna promenjiva koja je jednaka 1 ako je objekat postavljen na čvoru $j$ u periodu $t$
\item[$Y_{it}$] - Binarna promenjiva koja je jednaka 1 ako čvor $i$ u periodu $t$ je pokriven sa bar jednim objektom koji je postavljen na distanci manjoj od S u odnosu na čvor $i$.	
\end{description}


Za ovako definisane promenljive i parametre model glasi:
$$max \sum_{t=1}^{T} \sum_{i=1}^{I} a_{it} Y_{it} $$

Tako da važe uslovi:
$$Y_{it} \leq \sum_{j \in N_{i}} Xjt, \;\;\;\;\; i \in I, t \in T  $$
$$\sum_{t=1}^{T} \sum_{j=1}^{J} X_{jt} = p $$

Funkcija cilja služi da maksimizuje ukupnu potražnju. Prvi uslov onemogućava pokrivenost čvora u čijoj S blizini se ne nalazi bar 1 objekat. Dok drugi uslov pokazuje da se p objekata raspoređuje tokom T perioda. 

\section{Lokalna pretraga}

Lokalna pretraga pripada grupi S-metaheuristika, koje se zasnivaju na poboljšavanju vrednosti jednog
rešenja. Na početku algoritma se proizvoljno ili na neki drugi način generiše početno rešenje i izračuna vrednost njegove funkcije cilja. Vrednost najboljeg rešenja se najpre inicijalizuje na vrednost početnog. Zatim se
algoritam ponavlja kroz nekoliko iteracija. U svakom koraku se razmatra rešenje u okolini trenutnog. U zavisnosti od toga kako se definiše okolina mogu se dobiti dobri ilii loši rezultati. Ukoliko je vrednost njegove funkcije cilja bolja od vrednosti funkcije cilja trenutnog rešenja, ažurira se trenutno
rešenje. Takođe se, po potrebi, ažurira i vrednost najboljeg dostignutog rešenja. Algoritam se ponavlja
dok nije ispunjen kriterijum zaustavljanja. Kriterijum zaustavljanja može biti, na primer, dostignut maksimalan broj iteracija, dostignut maksimalan broj ponavljanja najboljeg rešenja, ukupno vreme izvršavanja,
itd. Lokalna pretraga se može prikazati sledećim pseudokodom


Pseudo-kod algoritma:
\begin{verbatim}
Lokalna pretraga()
Ulaz:

Izlaz:

begin
    Generisati početno rešenje s
    Inicijalizovati vrednost najboljeg rešenja f* <= f(s)
    dok nije ispunjen izabrani kriterijum zaustavljanja do
        izabrati proizvoljno rešenje s' u okolini s koje je validno
        ako f(s') < f(s) onda
            s <= s'
        ako f(s') < f* onda
            f* <= f(s')	
end
\end{verbatim}

U datoj verziji lokalne pretrage se vrši problem minimizacije, dok je u našem problemu potrebno maksimizovati ciljnu funkciju. To ćemo rešiti tako što ćemo prihvatati ona rešenja čija je vrednost funkcije cilja veća. Osnovna mana lokalne pretrage je u tome što ne nalazi globalna rešenja, i što kada uđe u lokanlni optimum iz njega ne može da izađe. Ovo se može poboljšati korišćenjem neke naprednije tehnike koja istražuje i rešenja koja nisu trenutno najbolja u nadi da se negde dalje krije bolje rešenje.

\section{Implementacija}

Projekat je implementiran korišćenjem programskog jezika Python. 
\subsection{Inicijalizacija}
\subsection{Izbor suseda}
\subsection{Rezultati}

%\begin{figure}
%	\centering
%	\includegraphics[scale=0.5]{resources/execution_comparison.png}
%	\caption{Poređenje efikasnosti}
%	\label{fig1}
%\end{figure}
\section{Primene}
Lokacijski problem maksimalnog pokrivanja je klasičan model koji je našao široku primenu u raznim oblastima nauke i privrede. Dinamička verzija ovog problema je relativno nova i još istraživanja je potrebno da bi došlo do njenih primena u većem broju situacija. Ovaj model nije neuobičajen u praksi i primenljiv je u mnogim slučajevima. Na primer može se koristiti za postavku policijskih patrola \cite{police}, premeštanje ambulanti \cite{ambulance} ili realokacija mesta za prvu pomoć kod prirodnih katastrofa.

\addcontentsline{toc}{section}{Literatura}
\appendix
\bibliography{seminarski} 
\bibliographystyle{plain}

\end{document}
